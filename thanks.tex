\chapter*{Agradecimentos}

Em toda caminhada longa e árdua, há momentos de tropeço e desânimo que só são superados quando alguém lhe estende a mão e lhe coloca novamente no caminho. No meu caso não foi diferente, portanto, tenho muitas mãos a agradecer.

Agradeço ao meu orientador, prof. Geraldo Zimbrão, por acreditar no meu potencial e pela orientação deste trabalho. Minha gratidão aos membros da banca pelo tempo e atenção devotados à análise desta dissertação. Em especial, ao prof. Carlos Mello, cujas ideias e conselhos foram fundamentais, não apenas em minha vida acadêmica, mas também em minha vida pessoal. Cadu, se assim me permite, meu muito obrigado.

Quero agradecer aos meus amigos do PESC, que, entre um café e outro, me deram dicas de leitura; apontaram meus erros; apludiram meus acertos; me deram caronas; me contaram piadas; escutaram pacientemente minhas angústias e alegrias. Braida, Duarte, Horta, Luis e Pedro, não sei se aguentaria sem vocês.

Não posso deixar de agradecer ao prof. Daniel Figueiredo que, desde da graduação, me motivou a entrar no mundo acadêmico e, durante o mestrado, sempre se mostrou solicito a escutar meus problemas e pensar em soluções. Agradeço também a todos os funcionários do PESC e da CAPES por proverem uma infraestrutura onde pude me apoiar.

Sem dúvida, o maior auxílio que tive foi o amor e suporte da minha família. Não só agradeço, como também dedico este trabalho a eles: meus pais, minha irmã, meus avós, meus tios e primos. Meu muito obrigado pelo carinho; pelas palavras de ternura; pela compreensão e pelos tantos ``vai dar tudo certo''.

Por último, mas não menos importante, agradeço à mão que me pôs de pé após as quedas mais dolorosas e angustiantes. Obrigado Deus, por tudo!