\chapter{Introdução}
\section{Motivação}

Há uma frase, de autor desconhecido, mas que ficou muito famosa, com o dizer: ``informação é o novo petróleo''. Apesar do conteúdo sensacionalista, não é possível negar que a veracidade da mesma vem se confirmando nas últimas décadas. De fato, nunca se produziu tanta informação em toda história e, ao mesmo tempo, nunca se dependeu tanto da mesma. As maneiras habituais de trabalho, estudo, locomoção, relacionamento e convívio foram completamente remodeladas, passando a ser fortemente informatizadas. Ou seja, os indivíduos estão cada vez mais dependentes de aparatos tecnológicos para executar tarefas antes consideradas básicas e triviais. Tal fenômeno chamou a atenção dos pesquisadores que o batizaram de \textit{Era da Informação} \citep{castells_information_1999}.

Existem muitas vantagens e desvantagens associadas à tal Era da Informação, assim como em tudo o mais. No entanto, uma de suas mais aclamadas vantagens está se transformando em uma desvantagem. Boa parte desta sobrecarga de informação se encontra atualmente na Internet, o que cria uma enorme oferta de opções para os usuários. A intuição comum normalmente associa variedade à satisfação, isto é, quanto mais opções um usuário tiver, consequentemente, mais satisfeito ele estará. Contudo, o que se observa na prática é que, frente a um demasiado numero opções, os usuários se angustiam, uma vez que o risco de fazer a escolha errada aumenta consideravelmente. Portanto, a exacerbada variedade de opções vem causando descontentamento ao invés de satisfação \citep{schwartz_paradox_2004}.

Este paradoxo torna-se mais visível quando consideramos o comércio eletrônico, também conhecido como \textit{e-commerce}. Uma das esferas da vivência humana mais afetada pela Era da Informação foi, sem dúvida, o comércio. Chega a ser difícil encontrar uma empresa que não oferte seus produtos \textit{online}. Por outro lado, a Internet deixou de ser apenas mais um canal de comunicação com o cliente e passou a ser o habitat natural de novos negócios, muitos deles comercializando bens intangíveis (e.g., \textit{software}, áudio, vídeo, etc.), que são facilmente armazenáveis. Some-se a isto o fato de que as lojas virtuais, quando ofertam bens tangíveis, isto é, produtos físicos em geral, não precisam tê-los em estoque no momento da venda. Tais fatores conferem aos \textit{e-commerces} uma enorme flexibilidade para montar seus catálogos, que podem atingir facilmente a ordem de milhões de itens.

Comprar, por sua vez, nunca foi uma ação tão simples e, ao mesmo tempo, tão complexa. Simples no sentido de que a revolução informacional acabou criando formas mais eficientes de pagamentos e, hoje, é possível literalmente comprar uma grande quantidade de itens com apenas um \textit{click}. Entretanto, escolher o que comprar se tornou uma tarefa complexa tendo em vista as inúmeras possibilidades oferecidas \textit{online}. O usuário comum se depara com uma infinidade de lojas virtuais, de toda parte do globo, cada uma com um catálogo praticamente ilimitado de itens. Não é atoa que estão surgindo diversos serviços voltados somente para auxiliar o usuário a encontrar o que deseja, como, por exemplo, buscadores específicos para produtos, comparadores de preço, atendimento virtual, entre outros.

Dentre as ferramentas que visam auxiliar o usuário durante a compra, nenhuma chamou tanto a atenção da indústria quanto os Sistemas de Recomendação (SR). Durante as últimas duas décadas, as grandes empresas da Internet investiram fortemente em tais sistemas, pois perceberam que o usuário se sente perdido em meio a tantas opções e, no fim, acaba por desistir da compra. Ao oferecer recomendações, o sistema pretende amenizar a dúvida do usuário, reduzindo o número de opções a se considerar. Ademais, a loja virtual passa a sensação de que compreende o usuário, ou seja, que sabe distinguir entre os itens que ele gosta e os que ele não gosta. 

%\abbrev{SR}{Sistema de Recomendação}

Além de, claro, aumentar a probabilidade de venda ao recomendar os itens que o usuário gosta (ou pelo menos que o site acredita que ele gosta), a sensação de ser atendido de forma personalizada estabelece, em muitos usuários, uma relação de confiança. Em outras palavras, SR não servem apenas para alavancar vendas, podem ser concebidos como uma forma de se estreitar o relacionamento com o cliente \citep{Schafer:1999:RSE:336992.337035}.

A academia, por sua vez, também investiu com veemência no aprimoramento das técnicas de recomendação. Em especial, a competição organizada pelo \textit{Netflix} trouxe muita atenção para este tema, uma vez que levou pesquisadores de todo o mundo a trabalharem em um algoritmo que fosse mais eficiente que o do próprio \textit{Netflix} \citep{Bennett07thenetflix}. Como resultado, diversos algoritmos foram propostos, a grande maioria se baseando em métodos de Aprendizado de Máquina (AM) e Data Mining (DM), o que trouxe importantes avanços para a área \citep{Bell:2007:LNP:1345448.1345465, ricci_recommender_2011}.

%\abbrev{AM}{Aprendizado de Máquina}
%\abbrev{DM}{Data Mining}

Em geral, as técnicas de recomendação podem ser divididas em três abrangentes categorias: Baseadas em Contexto (BC), Filtragem Colaborativa (FC) e Híbridas. No capítulo \ref{cap:fundamentacao}, iremos descrever em detalhes cada uma dessas categorias e as principais técnicas de cada uma. Por hora, é suficiente mencionar que as técnicas de FC se sobressaem às demais, principalmente em termos de desempenho e aplicabilidade. No entanto, elas sofrem com o problema do \textit{Arranque Frio} (AF), ou problema do \textit{Novo Usuário/Item}. Como muitas delas se baseiam em noções de similaridade entre usuários/itens, extraídas a partir das preferências dos usuários, não é possível calcular nenhuma recomendação para o novo usuário/item, visto que não se pode medir sua similaridade para com os demais usuários/itens \citep{su_survey_2009}.

%\abbrev{FC}{Filtragem Colaborativa}
%\abbrev{AF}{Arranque Frio}

Por esta razão, muitos trabalhos foram propostos na tentativa de amenizar o problema do AF em FC, sobretudo no tocante ao novo usuário \citep{Rashid:2002:GKY:502716.502737, Rashid:2008:LPN:1540276.1540302, elahi_rating_2011, elahi_system-wide_2011, Elahi:2014:ALS:2542182.2542195}. Tais propostas buscam aumentar o conhecimento do sistema sobre os novos usuários, solicitando aos mesmos que deem suas preferências (na forma de avaliações) para alguns itens previamente definidos. A este processo, que precede o cálculo das recomendações, damos o nome de \textit{Elicitação de Preferências}.

É importante delinear que a Elicitação de Preferências pode ser utilizada em contextos que não são necessariamente relacionados ao problema de AF. Infelizmente, na literatura, esses termos podem parecer sinônimos, uma vez que a grande maioria dos trabalhos envolvendo Elicitação de Preferências são voltados para contornar as deficiências geradas pelo AF. A fim de distinguir os cenários onde o processo de elicitação pode ser aplicado, iremos denominar o caso referente ao AF como \textit{Elicitação de Preferências para fins de Arranque Frio}.

\section{Proposta}

Avaliar não é um hábito comum para boa parte dos usuários, de forma que, na maioria das lojas virtuais, sabe-se apenas quem comprou o que. Informações sobre a preferência dos usuários é difícil de se obter, visto que não há um incentivo direto para que eles avaliem os itens que compraram. Claro que muitos contribuem com suas preferências na esperança de poder ajudar os que estão em dúvida sobre um determinado item; outros simplesmente porque gostam de deixar seu contentamento (ou seu desprezo) registrados como forma de gratidão (ou revolta) para com o fornecedor; e há ainda aqueles que entendem que o sistema depende das preferências para calcular as recomendações e, por isto, contribuem na esperança de melhorar a qualidade das mesmas.

Em todo o caso, a motivação para contribuir com as suas preferências possui sempre aspectos colaborativos, visando o ganho do sistema, ou de outros usuários, e nunca o ganho individual. Embora tenham ocorrido mudanças significativas, sobretudo na \textit{Web}, em direção a maior colaboração entre os usuários, o simples espírito altruísta parece não ser suficiente para mobilizar a maioria, que continua não contribuindo assiduamente. Enquanto boa parte dos trabalhos que se propõem a motivar os usuários acabam por enveredar pela via colaborativa, tentando maximizá-la \citep{Carenini:2003:TMC:604045.604052, ling_using_JCC4, Rashid:2006:MPD:1124772.1124915}, outros foram além e procuraram pensar em um modelo econômico que capturasse a maneira como os usuários dão suas preferências \citep{harper_economic_2005, market_avery_1999}. Em \citep{lee2013alleviating}, os autores procuram elicitar preferências da multidão, usando incentivos, na esperança de que isto melhore o desempenho das técnicas de FC. Como não é possível saber quais itens são avaliáveis, \citep{lee2013alleviating} acaba tratando da \textit{Elicitação de Preferências para fins de Arranque Frio} no contexto de multidão.

%Desta maneira, seria possível propor mecanismos de incentivo a ``produção de preferências'' similares aos incentivos governamentais a produção de bens de consumo, por exemplo. Todavia, ambas abordagens pecam no mesmo ponto, isto é, ambas tratam todas as preferências como possuindo a mesma importância, ou valor, para o SR.

Uma vez que tais sistemas possuem a informação de compra, isto é, sabem quem comprou o que, é razoável assumir que, se um usuário comprou um determinado item, logo, ele é capaz de avaliar este item. Esta suposição não se dá necessariamente de forma imediata, pois certos itens requerem uma quantidade significativa de tempo para serem ``consumidos'' (e.g., livros e filmes). Contudo, o que podemos assumir é que, dado que um usuário efetuou a compra de um item, passado um determinado intervalo de tempo, este usuário estará apto a dar sua preferência sobre o item em questão, seja ela boa ou ruim.

Ao solicitar que o usuário avalie um item que ele já comprou, a probabilidade de se obter uma preferência, em forma de avaliação, é muito maior do que quando o mesmo é solicitado a avaliar um item a esmo. Portanto, temos aqui o outro contexto onde o processo de Elicitação de Preferências se encaixa bem, isto é, elicitar as preferências sobre itens já comprados. É importante ressalvar que nada impede que o processo de elicitação seja aplicado concomitantemente tanto neste caso quanto no caso voltado para o AF. O que chama a atenção sobre este contexto é o fato dele ser até mais eficiente na aquisição de preferências do que no primeiro caso. A \textit{Amazon}, por exemplo, tenta persuadir seus clientes a contribuírem com suas preferências enviando \textit{emails}, tempos após a compra, contendo os itens comprados e a opção de avaliá-los \citep{linden_amazon_2003}.

No só a \textit{Amazon}, mas praticamente todas as lojas virtuais procuram elicitar as preferências a respeito dos itens comprados pelo usuário, este fato, em si, não é nenhuma novidade. Geralmente, o processo de elicitação se dá através do envio de \textit{emails} ou alguma outra forma de notificação. Todavia, se os \textit{e-commerces} solicitarem que cada usuário avalie todos os itens já comprados, poderão gerar um excesso de notificações que causará incômodo aos usuários. Infelizmente, na prática, é exatamente isto o que acontece e, o pior, os usuários mais incomodados são justamente aqueles que mais compram, ou seja, os melhores clientes. Este erro ocorre devido a uma suposição muito questionável, a de que todas as preferências contribuem da mesma forma para o SR. É muito plausível que certas preferências possuam maior impacto no desempenho do SR que outras, assim poderíamos obter um ganho equivalente, ou até mesmo superior, solicitando menos notificações aos clientes, o que lhes pouparia paciência.

Tanto neste caso como no caso referente ao AF, há um problema em comum no tocante a como selecionar os itens que devem ser avaliados por cada usuário. Nos trabalho voltados para a \textit{Elicitação de Preferências para fins de Arranque Frio}, os itens a se avaliar são escolhidos através de estratégias de Aprendizado Ativo (AA). Tais técnicas surgiram como uma subárea dentro de AM e visam selecionar as ``melhores'' instâncias de modo a construir o menor e mais eficaz conjunto de treinamento possível. Ou seja, estratégias de AA possibilitam que os modelos de AM sejam treinados de maneira mais rápida, tornando-os mais precisos com menos esforço de treinamento. No cenário referente a SR, a técnica de recomendação e as preferências podem ser considerados modelo de AM e instâncias de treinamento, respectivamente. Um apanhado geral sobre as diversas estratégias de AA pode ser encontrado no capítulo \ref{cap:fundamentacao}

%\abbrev{AA}{Aprendizado Ativo}

Supondo que o \textit{e-commerce} saiba quais são os $N$ itens que, se avaliados pelo usuário, irão trazer o maior ganho possível para o SR, em termos de desempenho, o mesmo pode oferecer ao usuário um incentivo individual a fim de garantir que este contribua com suas preferências. Como exemplo de incentivo pessoal, podemos citar desconto em futuras compras; pontos a serem trocados pro brindes ou vales; \textit{upgrade} no tipo de conta; entre outros. É importante frisar que a escolha dos incentivos que serão implantados e como os mesmos serão implantá-los foge a nossa \textit{expertise}, visto que se trata de um problema de administração ou de economia. Entretanto, a tarefa de encontrar os $N$ itens a serem avaliados por cada usuário pode ser atacada através de estratégias de AA. Este trabalho propõe portanto o emprego de tais estratégias a fim de se encontrar os $N$ itens mais importantes para cada usuário avaliar. A importância dessas preferências para o desempenho do SR deve ser tamanha a ponto do sistema estar disposto a oferecer incentivos individuais ao usuário pelas mesmas. Convencionamos chamar este problema de \textit{Elicitação de Preferências para fins de Incentivo}.

Mais ainda, uma das principais deficiências das estratégias de AA é o fato de que todas se baseiam em suposições sobre os dados, i.e., heurísticas. Quando estas suposições não se verificam na prática, as estratégias acabam por gerar um conjunto de treinamento enviesado, algo que, ao invés de impulsionar, prejudica o desempenho do sistema. Em suma, a qualidade das recomendações depende diretamente da estratégia empregada pelo SR, logo é vital que os projetistas saibam escolher aquela que trará o maior ganho para o sistema. 

Para não cairmos na problemática relacionada ao viés, propomos o emprego de uma nova estratégia. A batizada \textit{Estratégia Livre de Viés} ainda é desconhecida pela literatura relacionada a SR, sendo este trabalho sua primeira utilização em tal contexto. Como veremos no capítulo \ref{cap:estrategia-livre-vies}, esta estratégia busca selecionar os itens de forma a aproximar a Função de Distribuição de Probabilidade (FDP) dos mesmos, garantindo que o conjunto de treinamento gerado é o que melhor representa a totalidade dos dados, i.e., sem viés. 

Realizamos uma extensa comparação entre as estratégias de AA, em duas bases de dados consideradas clássicas na literatura. Ao todo, foram comparadas 17 estratégias, incluindo a \textit{Estratégia Livre de Viés}, e verificou-se que esta última, em ambas as bases, supera as demais. O desempenho de cada estratégia foi analisado no capítulo \ref{cap:resultados}, onde explicamos o motivo do sucesso ou fracasso das mesmas considerando as características das bases de dados. Mesmo havendo estratégias que possuem desempenho próximo a \textit{Estratégia Livre de Viés}, acreditamos ter encontrado uma excelente opção para todos os tipos de bases.

\section{Contribuições}

As contribuições deste trabalhos podem ser resumidas aos itens abaixo:

\begin{itemize}

\item Introdução do problema referente a \textit{Elicitação de Preferências para fins de Incentivo}, que não foi devidamente explorado na literatura até então.
\item Emprego de uma estratégia nunca antes utilizada no contexto de SR, a \textit{Estratégia Livre de Viés}, que garante produzir um conjunto de treinamento sem viés.
\item Realização do experimento mais completo da literatura, envolvendo ao todo 17 estratégias, incluindo a \textit{Estratégia Livre de Viés}.
\item Analise do desempenho, seja ele bom ou ruim, de cada estratégia, levando em consideração as características da base de dados.
\end{itemize}

\section{Organização}

Esta dissertação está organizada em 6 capítulos, dos quais este é o primeiro. No capítulo \ref{cap:fundamentacao}, iremos descrever os principais conceitos teóricos envolvidos, além de mencionar os trabalhos relacionados ao tema. No capítulo \ref{cap:estrategia-livre-vies}, detalharemos como funciona a \textit{Estratégia Livre de Viés} e como a mesma foi adaptada ao problema de SR. No capítulo \ref{cap:metodologia}, apresentamos a metodologia que guiou nossos experimentos, bem como as demais estratégias comparadas. No capítulo \ref{cap:resultados}, expomos os resultados obtidos e iniciamos uma investigação tentando encontrar os motivos por trás do sucesso ou fracasso de cada estratégia. Finalmente, no capítulo \ref{cap:conclusao}, tiramos as devidas conclusões e apontamos possíveis trabalhos futuros.